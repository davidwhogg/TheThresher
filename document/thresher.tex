\documentclass[12pt,preprint]{aastex}

% has to be before amssymb it seems
\usepackage{color,hyperref}
\definecolor{linkcolor}{rgb}{0,0,0.5}
\hypersetup{colorlinks=true,linkcolor=linkcolor,citecolor=linkcolor,
            filecolor=linkcolor,urlcolor=linkcolor}

\usepackage{url}
\usepackage{algorithmic,algorithm}
\usepackage{amssymb,amsmath}
\usepackage{graphicx}

\newcommand{\arxiv}[1]{\href{http://arxiv.org/abs/#1}{arXiv:#1}}

\newcommand{\project}[1]{{\sffamily #1}}
\newcommand{\TheThresher}{\project{The~Thresher}}
\newcommand{\LCOGT}{\project{LCOGT}}
\newcommand{\AstraLux}{\project{AstraLux}}
\newcommand{\Python}{\project{Python}}
\newcommand{\numpy}{\project{numpy}}
\newcommand{\github}{\project{GitHub}}
\newcommand{\pip}{\project{pip}}
\newcommand{\paper}{\emph{Article}}
\newcommand{\license}{GNU General Public License v2}

\newcommand{\documentname}{\textsl{Article}}

\newcommand{\foreign}[1]{\emph{#1}}
\newcommand{\etal}{\foreign{et\,al.}}
\newcommand{\etc}{\foreign{etc.}}

\newcommand{\Fig}[1]{Figure~\ref{fig:#1}}
\newcommand{\fig}[1]{\Fig{#1}}
\newcommand{\figlabel}[1]{\label{fig:#1}}
\newcommand{\Tab}[1]{Table~\ref{tab:#1}}
\newcommand{\tab}[1]{\Tab{#1}}
\newcommand{\tablabel}[1]{\label{tab:#1}}
\newcommand{\Eq}[1]{Equation~(\ref{eq:#1})}
\newcommand{\eq}[1]{\Eq{#1}}
\newcommand{\eqlabel}[1]{\label{eq:#1}}
\newcommand{\Sect}[1]{Section~\ref{sect:#1}}
\newcommand{\sect}[1]{\Sect{#1}}
\newcommand{\App}[1]{Appendix~\ref{sect:#1}}
\newcommand{\app}[1]{\App{#1}}
\newcommand{\sectlabel}[1]{\label{sect:#1}}
\newcommand{\Algo}[1]{Algorithm~\ref{algo:#1}}
\newcommand{\algo}[1]{\Algo{#1}}
\newcommand{\algolabel}[1]{\label{algo:#1}}

% math symbols
\newcommand{\dd}{\mathrm{d}}
\newcommand{\like}{\mathscr{L}}
\newcommand{\bvec}[1]{\boldsymbol{#1}}
\newcommand{\given}{\,|\,}
\newcommand{\transpose}[1]{{#1}^{\mathsf{T}}}
\newcommand{\unit}[1]{\,\mathrm{#1}}
\DeclareMathOperator*{\argmax}{arg\,max}

% Parameters
\newcommand{\data}{\ensuremath{D}}
\newcommand{\scene}{\ensuremath{\Sigma}}
\newcommand{\kernel}{\ensuremath{K}}
\newcommand{\psf}{\ensuremath{\psi}}
\newcommand{\dpsf}{\ensuremath{\tilde{\psi}}}
\newcommand{\dvec}{\ensuremath{d}}
\newcommand{\evec}{\ensuremath{e}}
\newcommand{\svec}{\ensuremath{s}}
\newcommand{\smat}{\ensuremath{S}}
\newcommand{\pvec}{\ensuremath{p}}
\newcommand{\pmat}{\ensuremath{P}}

\newcommand{\dfmplot}[1]{%
\begin{center}%
%    \includegraphics[width=\textwidth]{#1}%
\end{center}%
}
\newcommand{\threshplot}[1]{%
\begin{center}%
% \includegraphics[width=0.8\textwidth,trim=21cm 11cm 4cm 11.8cm,clip=true]{#1}%
\end{center}%
}

\begin{document}

\title{\TheThresher: Lucky imaging without all the waste}

\author{%
    Daniel~Foreman-Mackey\altaffilmark{\ref{CCPP},\ref{email}},
    David~W.~Hogg\altaffilmark{\ref{CCPP},\ref{MPIA}}
    \& others TBD
}

\newcounter{address}
\setcounter{address}{1}
\altaffiltext{\theaddress}{\stepcounter{address}\label{CCPP} Center
  for Cosmology and Particle Physics, Department of Physics, New York
  University, 4 Washington Place, New York, NY 10003}
\altaffiltext{\theaddress}{\stepcounter{address}\label{email} To whom
  correspondence should be addressed: \texttt{danfm@nyu.edu}}
\altaffiltext{\theaddress}{\stepcounter{address}\label{MPIA}
  Max-Planck-Institut f\"ur Astronomie, K\"onigstuhl 17, D-69117
  Heidelberg, Germany}

\begin{abstract}
    In traditional lucky imaging (TLI), many images are taken with a
    fast camera, and all but the best percent or so of the data are
    discarded in the final shift-and-add combined image.  Here we
    present an alternative image analysis
    pipeline---\TheThresher---for these kinds of data, based on online
    multi-frame blind deconvolution.  It makes use of all available
    data to obtain the best estimate of the astronomical scene in the
    context of reasonable computational limits.  Because it uses the
    full data set, \TheThresher\ outperforms TLI in signal-to-noise;
    because it accounts for the individual-frame PSFs, it does this
    without loss of angular resolution.  We demonstrate effectiveness
    on artificial data but also real data from the \LCOGT\ and
    \AstraLux\ lucky imaging experiments.  Source code available at
    \url{http://danfm.ca/thresher} under the \license.
\end{abstract}

\keywords{%
    methods: data analysis ---
    methods: numerical ---
    methods: statistical
}

\clearpage

\section{Introduction}

High-resolution and high-contrast imaging has seen great success in
the last decades, with the \project{Hubble Space Telescope}, natural
and artificial guide-star adaptive optics, coronography, speckle
imaging, lucky imaging, and various new data analysis techniques.
With ground-based telescopes, very sophisticated hardware and/or
software is required if one hopes to obtain high angular resolution.
This is because atmospheric turbulence causes high frequency (in time
and space) variations in the point spread function (PSF) that quickly
degrade the angular resolution of any imaging. The PSF of a standard
long exposure suitable for astronomy will be smooth and much broader
than the ideal diffraction limit. This is because it is produced by
the time average over atmospheric variations, which create randomly
wandering and fluctuating speckles. The standard post-processing
techniques for dealing with this involve obtaining a large number of
extremely short exposures---on the timescale of the atmospheric
variations. These short exposures individually have low
signal-to-noise and very complicated PSFs, but they provide strong
constraints on the diffraction limited scene when many images are used
in concert.

The idea of exploiting the high resolution information in short
exposures was first suggested by \citet{labeyrie} and there has been a
rich literature on this topic since. In particular, a very popular
technique which we will call \emph{traditional lucky imaging}
\citep[TLI;][]{law} is based on Fried's derivation of the probability
of obtaining a frame with unusually ``lucky'' seeing when many images
are taken quickly. That is, occasionally---just by chance---the
wavefront distortions from the atmosphere will come close to canceling
(or really canceling the imperfections in your real telescope).  These
best frames are identified, shifted, and co-added, with the rest of
the data relegated to the dustbing.  TLI is very popular because it is
simple to understand and implement, it is computationally tractable,
and it can be used with very inexpensive equipment.  As a technique,
however, it is not \emph{really} inexpensive because it results in a
substantial amount of wasted data; typical TLI results are based on
only the best \emph{percent} of the acquired data.

The work presented here was motivated by the (correct) feeling
that---treated correctly---there is no way that the traditionally
discarded 99~percent of a TLI imaging set could be \emph{negatively}
useful in constraining the astronomical scene!  In the context of TLI,
the fundamental reason that throwing away data \emph{helps} is that
the core data analysis step is shift-and-add co-addition of the
imaging, which (though widely used in astronomy) is not justifiable
when the point-spread function is varying rapidly.

The pipeline we present here---\TheThresher---is a new implementation
of an old idea.  It is a special case of a broad class of algorithms
called \emph{blind deconvolution} \citep{ayers}.  In particular, it is
an implementation of the multi-frame online blind deconvolution (MOBD)
image analysis method of \citet{hirsch}. In this \documentname, we
describe the theory behind it and details of our particular
implementation.  Most importantly, we demonstrate the significant
improvements over TLI made possible by this technique using real data
from the \LCOGT\ and \AstraLux\ fast imaging cameras.

After the MOBD method on which the \TheThresher\ is directly based,
the most similar methods for fast imaging are from the speckle imaging
community.  Speckle imaging is usually analyzed via Fourier methods
(NEED CITATIONS, INCLUDING HOLOGRAPHIC), but it is not dissimilar to
blind deconvolution, since the fourier methods can be thought of as
creating models of the point-spread function.  The principal
conceptual difference between \TheThresher\ and most of these methods
is that \TheThresher\ is optimizing a \emph{justified scalar
  objective,} which itself is an approximation to a likelihood
function for the astronomical scene given the full set of imaging
data.  An additional important difference with most of the prior work
is that \TheThresher\ does not rely on the operator supplying
point-spread function information, nor does it rely on having
identified point sources in the imaging.

\section{Imaging} \sectlabel{imaging}

From our perspective, an image read out by a real camera (with, say,
square pixels in a focal-plane array) is a noisy sampling of the
intensity field, convolved with some kind of point-spread function.
Importantly, if you want to think of the image as a pure
\emph{sampling} of a convolved intensity field---and trust us, you
do---then the point-spread function that convolves the intensity field
should be not the pure atmospheric PSF, nor even the
instrument-convolved atmospheric PSF\@.  It should be the
\emph{pixel-convolved} PSF \emph{at the focal plane}.  From here on,
whenever we mention or use the PSF, we mean \emph{always} the
pixel-convolved PSF at the focal plane.  This choice may seem strange,
but when this pixel-convolved PSF is used, the pixel values are
delta-function samples of the convolved intensity field, reducing
enormously synthetic-image computation.  Furthermore, if the
non-pixel-convolved PSF is smooth and well-sampled, the
pixel-convolved PSF is \emph{also} smooth, so there are no significant
numerical losses or approximations incurred by making this choice.

% % Maybe I don't get it but this seems not totally needed...
% It pains us to point out that the PSF, as it is usually conceived, is
% actually \emph{correlated} not \emph{convolved} with the true scene.
% That is a choice about what the PSF is.  In what follows we will use
% the word ``convolve'' in conformity with usual practice in astronomy,
% but the way we show the PSFs in the figures, it is probably
% ``correlate'' that we are really doing.

To make our ideas about imaging concrete, we can represent the model
that will be used throughout this paper for imaging data as a
convolution
\begin{equation}\eqlabel{convimg}
    \data_n = \psf_n \ast \scene + E_n \quad,
\end{equation}
where $\data_n$ is the data image---one of $N$ noisy $M$-pixel
images---with index $1<n<N$, $\psf_n$ is the pixel-convolved PSF
appropriate for image $n$, $\ast$ represents the convolution
operation, $\scene$ is the ``true'' intensity field (``scene'') above
the atmosphere, and $E_n$ is the noise contribution to image $n$.
Since convolution is a linear operation, it can be written in matrix
form. This is especially easy to visualize in this context where all of
the convolutions are discrete. Casting \eq{convimg}---our
\emph{generative model} for an astronomical image---as a linear system
has significant computational benefits and there are many techniques
available for solving such an system. Furthermore, following
\citet{hirsch}, we can write the model in two equivalent ways:
\begin{eqnarray}\displaystyle
\dvec_n &=& \pmat_n \cdot \svec + \evec_n \eqlabel{linimg1}
\\
\dvec_n &=& \smat \cdot \pvec_n + \evec_n \eqlabel{linimg2}
\quad ,
\end{eqnarray}
where now $\dvec_n$ is the original data image $\data_n$ reformatted as a
one-dimensional length-$M$ column vector, $\pmat_n$ is a $W\times M$
sparse matrix that contains $K$ independent values given by the entries
in $\psf_n$, $s$ is a length-$W$ column vector representing the true
scene, $\evec_n$ is a length-$M$ column vector of noise contributions to
image $n$, $\smat$ is a $K\times M$ dense matrix that contains $W$
independent values representing the scene, and $\pvec_n$ is a length-$K$
column vector representing the PSF\@. In \eq{linimg1} and \eq{linimg2},
the transformations
$\psf_n \to \pmat_n$ and $\scene \to \smat$ are illustrated in \fig{index}.
The idea is that if the image data are unwrapped into a
one-dimensional vector, the linear convolution operation can always be
represented as a matrix operation acting on another vector, and there
is a choice of whether to see the PSF as the matrix and the scene as
the vector, or vice versa.

\begin{figure}[!htbp]
    \dfmplot{index_gymnastics.pdf}
    \caption{Demonstration of the index gymnastics needed to convert between
        images and the matrix representation of convolution.\figlabel{index}}
\end{figure}

In real astronomical applications, neither the PSF nor the true scene
is known \foreign{a priori}. If our interest is the true scene---and
it usually is---then this is a special case of what's known in
the computer science literature as \emph{blind deconvolution}. We
want to explain the data as being produced by something fundamental of
interest convolved with instrumental resolution that is not known in
advance and also of no particular interest in itself.

One small but perhaps not insignificant issue with the model expressed
here is that it posits the existence of a ``true scene''.  For
technical reasons---related to the finiteness of any real data
stream---it is better to think of this ``true image'' as really an
instrument for making finite-resolution predictions.  It is not going
to be an accurate representation of the scene we would see with an
implausible infinitely large telescope!  It is a representation of the
intensity field that is only to be used within the context of
convolution with a finite PSF\@.  Similar issues arise in radio astronomy
when interferometric data are ``cleaned''.

Now, with an error model---a probabilistic description of how the
$E_n$ image (or equivalently, $\evec_n$ vector) is generated---we can
write down a likelihood function or a probability for the data $\data_n$
(or equivalently $\dvec_n$) given the model parameters $\pvec_n$ and
$\svec$.  For example, if the $M$ per-pixel noise contributions are
(independently) drawn from Gaussians with zero means then this likelihood
has the incredibly simple form
\begin{equation}\eqlabel{lnlike}
\ln p(\data_n \given \psf_n, \scene) = Q - \frac{1}{2}\,\chi^2_n
\end{equation}
where $Q$ is a normalization constant and
\begin{equation}\eqlabel{chi2}
    \chi^2_n \equiv \transpose{[d_n - P_n \cdot s]} \cdot C^{-1}
    \cdot [d_n - P_n \cdot s] \quad.
\end{equation}
In \eq{chi2}, $C^{-1}$ is the diagonal $M \times M$
matrix with the per-pixel inverse variance on the diagonal. If the
noise was not independent between pixels, there would be off-diagonal
terms in $C^{-1}$.

In principle, it is even possible to write down a prior over possible
scenes and PSFs. Combining these priors with \eq{lnlike} yields the
posterior probability distribution function (PDF) given the full dataset.
Then, it is theoretically possible to marginalize this posterior PDF
over all possible PSFs and recover the distribution over possible
scenes $\scene$. That would be just about the best we could possibly do
in this problem, for very general reasons.
Unfortunately, even if we apply unrealistically restrictive priors on
image space, any posterior PDF over true scenes will be a function in
$> 10^4$ dimensions even for restrictively small patches of data.
Current techniques and computational power are not yet sufficient
to properly solve this problem. Instead, we restrict ourselves to the
\emph{maximum a posteriori} (MAP) optimization task. Instead of
marginalizing over PSFs, MAP inference involves finding the values of
$\scene$ and $\{\psf_n\}$ that maximize the posterior PDF
\begin{equation}\eqlabel{argmax1}
    \scene^*, \{\psf^*_n\} = \argmax_{\scene, \{\psf_n\}}
        \prod_n p ( \scene, \psf_n \given \data_n) \quad.
\end{equation}
If we notice that $\log x$ is a monotonic function of $x$ and neglect
the various irrelevant normalization constants, \eq{argmax1} can be
equivalently posed
\begin{equation}\eqlabel{argmax2}
    \scene^*, \{\psf^*_n\} = \argmax_{\scene, \{\psf_n\}}
        \sum_n \left [ -\frac{1}{2} \, \chi^2
            + \log p (\scene) + \log p (\psf_n)
        \right ] \quad ,
\end{equation}
where $p (\scene)$ and $p(\psf)$ are our prior probability functions
over possible scenes and PSFs respectively. One could assign equal
probability to all scenes and PSFs and find the maximum-likelihood (ML)
solution but in practice, there are significant degeneracies and
instabilities in the ML problem and it is useful to use priors---or
regularizations, as they are often reffered to in the machine learning
literature---to keep the problem tractable. The priors that we choose
are
\begin{equation}\eqlabel{psf-prior}
    \log p(\psf_n) = - \gamma \, \left ( 1 - \sqrt{\pvec_n \cdot \pvec_n}
        \right )^2 \quad,
\end{equation}
and
\begin{equation}\eqlabel{scene-prior}
    \log p(\scene) = - \lambda \, \svec \cdot \svec \quad,
\end{equation}
where $\gamma$ and $\lambda$ are free parameters that can be adjusted
to improve performance. Intuitively, \eq{psf-prior} prefers PSFs that sum
to approximately unity and \eq{scene-prior} forces the scene towards zero
unless the data is informative. \Eq{scene-prior} is an example of the
popular L2-regularization technique and it has the effect of reducing
overfitting when the model has many parameters.

It is also standard practice to constrain both $\scene$ and $\psf$ to only
have non-negative values. This is usually justified using ``physical''
arguments. At first sight, this might seem natural butm in fact, it is not
correct. While it is reasonable to force each PSF to be non-negative, it
is likely that the faintest sources in the full inferred scene are
below the detection threshold in any single observation but should be
easily detectable in final scene. These sources will be very strongly
affected by an non-negativity constraint because they might---in the units
of the inferred scene---have \emph{negative flux}! It is, therefore,
irresponsible to apply any non-negativity constraint to $\scene$ during
the optimization procedure.

\section{Optimization}

To perform the optimization in \eq{argmax2}, we use a variation of the
online algorithm called \emph{stochastic gradient}. This algorithm is
called online because it operates on a single data point at a time
meaning that the full dataset never needs to be loaded into memory
simultaneously. This allows the method to scale well to very large
datasets. If we call the set (or vector) of parameters that we are
optimizing $w \equiv \{\scene, \psf_n\}$ and the objective function
\begin{equation}
    \ell (w) = \sum_n \ell_n (w_n) = \sum_n \left [ -\frac{1}{2} \, \chi^2
            + \log p (\scene) + \log p (\psf_n)
        \right ]
\end{equation}
then a traditional bulk
gradient descent method performs iterative updates to the parameters
of the form
\begin{equation}\eqlabel{bulkgrad}
    w^{(t+1)} \gets w^{(t)} - \alpha_t \, \nabla_w \ell (w^{(t)})
    = w^{(t)} - \alpha_t \, \sum_n \nabla_{w_n} \ell_n (w_n^{(t)}) \quad,
\end{equation}
where $\alpha_t$ is a step size that can be tuned for efficiency and
robustness. The problem with \eq{bulkgrad} is that it involves optimizing
using \emph{all of the data simultaneously}. The stochastic gradient
method relies on the linearity of the gradient to instead perform
updates of the form
\begin{equation}
    w^{(t+1)} \gets w^{(t)} - \alpha_t \, \nabla_{w_m^{(t)}}
        \ell_{m} (w_m)
\end{equation}
where $m$ indicates a single data point that is different for each
iteration.

In the notation of the previous section, this can be rewritten as
\begin{eqnarray}
    \svec^{(t+1)}_m &\gets& \svec^{(t)}_m - \alpha_t \,
    \frac{\dd \ell_n (w_n^{(t)})}{\dd \svec_m^{(t)}} \quad, \quad
    \forall m = 1,\ldots,W \\
    \pvec^{(t+1)}_k &\gets& \pvec^{(t)}_k - \alpha_t \,
    \frac{\dd \ell_n (w_n^{(t)})}{\dd \pvec_k^{(t)}} \quad, \quad
    \forall k = 1,\ldots,K \quad.
\end{eqnarray}

\section{Light Deconvolution}

In their treatment of the deconvolution problem, \citet{magain} note that
standard deconvolution techniques produce ring artifacts around point
sources because the deconvolved scene violates the sampling theorem (CITE).
This is because the \emph{perfect} deconvolution of a point source is
essentially the Dirac $\delta$-function. A $\delta$-function can only be
properly sampled with \emph{infinite} resolution and when it is resampled
on a coarser grid, the brightness profile takes the form
$\sim \frac{\sin r}{r}$. This effect is related to our argument that it
is irrelevant to even imagine the \emph{true} scene. Instead, the best
that one can do is to deconvolve to a system with larger but finite
resolution.

\citet{magain} go on to suggest a solution to this problem. Instead of
trying to infer the scene at infinite resolution, it is better to rewrite
\eq{convimg} as
\begin{equation}
    \data_n = \dpsf_n \ast \kernel \ast \scene + E_n
\end{equation}
where $\dpsf_n$ is the ``deconvolved PSF'' (dPSF) and \kernel\ is the
PSF of the final scene. Then, the inference task is to infer
\begin{equation}
    \tilde{\scene} \equiv \kernel \ast \scene
\end{equation}
instead of \scene\ itself. This is achieved by re-writing \eq{linimg2} as
\begin{equation}
    \dvec_n = \tilde{\smat} \cdot \pvec_n + \evec_n
\end{equation}
where $\tilde{\smat}$ is related to $\tilde{\scene}$ in the same way that
\smat\ is related \scene.

\section{The Algorithm}

The method used to analyze a stack of images from a TLI run is as follows:
\begin{enumerate}
    \item{Run TLI on the stack to pre-align the data and co-add the best
        $\sim 1\%$ to get an initial guess at the scene $\svec^{(0)}$.}
    \item{For each image $n = 1, \ldots, N$,
\begin{itemize}
    \item{Infer the PSF $\pvec_n$ using \eq{linimg2} and the current best
        scene $\svec^{(n-1)}$ convolved with the target PSF \kernel,}
    \item{Infer the implied scene $\hat{\svec}$ using \eq{linimg1} and
        $\pvec_n$, and}
    \item{Apply constraints (sky subtraction, non-negativity, \etc)
            and the new estimate for the scene
            \begin{equation}\eqlabel{ourupdate}
                \svec^{(n)} \gets \svec^{(n-1)} - \alpha_n \,
                (\svec^{(n-1)} - \hat{\svec}) \quad.
            \end{equation}
        }
\end{itemize}
        }
\end{enumerate}

The update in \eq{ourupdate} is similar to the one suggested by
\citet{hirsch} with an extremely good choice of step-size. \citet{hirsch}
don't ever fully solve the linear least-squares problem and instead only
use an approximation to the solution. Our method is more computationally
intensive but it converges faster since it is more accurate.

\section{Experiments \& Results}

We ran both TLI and \TheThresher\ on a stack of 30000 images of a
trinary star system from \LCOGT\@. We show the results using 10, 30, 100,
300, 1000, 3000 images stretched to roughly the same
grayscale in all images.

\begin{figure}[!htbp]
    \dfmplot{tli10.png}
    \caption{The results of co-adding the top 10 best images as determined
        by TLI\@. The image on the right is the same data as the figure on the
        left with a harder stretch.\figlabel{tli:10}}
\end{figure}

\begin{figure}[!htbp]
    \threshplot{thresh10.png}
    \caption{The results of running \TheThresher\ on 10 images.
        The image on the right is the same data as the figure on the
        left with a harder stretch.\figlabel{thresh:10}}
\end{figure}

\begin{figure}[!htbp]
    \dfmplot{tli30.png}
    \caption{The same as \fig{tli:10} for the top 30 best images.
            \figlabel{tli:30}}
\end{figure}

\begin{figure}[!htbp]
    \threshplot{thresh30.png}
    \caption{The same as \fig{thresh:10} for the top 30 best images.
            \figlabel{thresh:30}}
\end{figure}

\begin{figure}[!htbp]
    \dfmplot{tli100.png}
    \caption{The same as \fig{tli:10} for the top 100 best images.
            \figlabel{tli:100}}
\end{figure}

\begin{figure}[!htbp]
    \threshplot{thresh100.png}
    \caption{The same as \fig{thresh:10} for the top 100 best images.
            \figlabel{thresh:100}}
\end{figure}

\begin{figure}[!htbp]
    \dfmplot{tli300.png}
    \caption{The same as \fig{tli:10} for the top 300 best images.
            \figlabel{tli:300}}
\end{figure}

\begin{figure}[!htbp]
    \threshplot{thresh300.png}
    \caption{The same as \fig{thresh:10} for the top 300 best images.
            \figlabel{thresh:300}}
\end{figure}

\begin{figure}[!htbp]
    \dfmplot{tli1000.png}
    \caption{The same as \fig{tli:10} for the top 1000 best images.
            \figlabel{tli:1000}}
\end{figure}

\begin{figure}[!htbp]
    \threshplot{thresh1000.png}
    \caption{The same as \fig{thresh:10} for the top 1000 best images.
            \figlabel{thresh:1000}}
\end{figure}

\begin{figure}[!htbp]
    \dfmplot{tli3000.png}
    \caption{The same as \fig{tli:10} for the top 3000 best images.
            \figlabel{tli:3000}}
\end{figure}

\begin{figure}[!htbp]
    \threshplot{thresh3000.png}
    \caption{The same as \fig{thresh:10} for the top 3000 best images.
            \figlabel{thresh:3000}}
\end{figure}

\acknowledgements It is a pleasure to thank
  Mike Blanton (NYU),
  Adam Bolton (Utah),
  Brendon Brewer (UCSB),
  Stefan Harmeling (T\"ubingen),
  Michael Hirsch (UCL),
  Phil Marshall (Oxford), and
  Bernhard Sch\"olkopf (T\"ubingen)
for contributions and comments.  We also single out
  Dustin Lang (CMU) and
  Robert Lupton (Princeton)
for special thanks; DWH wouldn't understand anything about imaging if
it weren't for the many years of discussions.  DWH and DFM were
partially supported by NASA (grant NNX12AI50G) and the NSF
(grant IIS-1124794).  This project made use of the NASA
\project{Astrophysics Data System}, and code in the the
\project{numpy}, \project{scipy}, and \project{matplotlib} open-source
projects.  All the code used in this paper is available at
\url{http://danfm.ca/thresher}.

\begin{thebibliography}{}\raggedright

\bibitem[Ayers\ \&\ Dainty(1988)]{ayers}
    Ayers, G. R., \& Dainty, J. C. 1988, Opt. Lett., 13, 547

\bibitem[Fried(1978)]{fried}
    Fried, D. L. 1978, J. Opt. Soc. Amer., 86

\bibitem[Hirsch\ \etal(2011)]{hirsch}
Hirsch, M., Harmeling, S., Sra, S., \& Sch{\"o}lkopf, B.\ 2011, \aap, 531, A9

\bibitem[Labeyrie(1970)]{labeyrie} Labeyrie, A.\ 1970, \aap, 6, 85

\bibitem[Law\ \etal(2006)]{law}
    Law, N.~M., Mackay, C.~D., \& Baldwin, J.~E.\ 2006, \aap, 446, 739

\bibitem[Magain\ \etal(1998)]{magain} Magain, P., Courbin, F.,
    \& Sohy, S.\ 1998, \apj, 494, 472



\end{thebibliography}

\clearpage
\appendix

\end{document}
